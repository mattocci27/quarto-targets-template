% Options for packages loaded elsewhere
\PassOptionsToPackage{unicode}{hyperref}
\PassOptionsToPackage{hyphens}{url}
\PassOptionsToPackage{dvipsnames,svgnames,x11names}{xcolor}
%
\documentclass[
  12pt,
  letterpaper,
  DIV=11,
  numbers=noendperiod]{scrartcl}

\usepackage{amsmath,amssymb}
\usepackage{lmodern}
\usepackage{iftex}
\ifPDFTeX
  \usepackage[T1]{fontenc}
  \usepackage[utf8]{inputenc}
  \usepackage{textcomp} % provide euro and other symbols
\else % if luatex or xetex
  \usepackage{unicode-math}
  \defaultfontfeatures{Scale=MatchLowercase}
  \defaultfontfeatures[\rmfamily]{Ligatures=TeX,Scale=1}
\fi
% Use upquote if available, for straight quotes in verbatim environments
\IfFileExists{upquote.sty}{\usepackage{upquote}}{}
\IfFileExists{microtype.sty}{% use microtype if available
  \usepackage[]{microtype}
  \UseMicrotypeSet[protrusion]{basicmath} % disable protrusion for tt fonts
}{}
\makeatletter
\@ifundefined{KOMAClassName}{% if non-KOMA class
  \IfFileExists{parskip.sty}{%
    \usepackage{parskip}
  }{% else
    \setlength{\parindent}{0pt}
    \setlength{\parskip}{6pt plus 2pt minus 1pt}}
}{% if KOMA class
  \KOMAoptions{parskip=half}}
\makeatother
\usepackage{xcolor}
\usepackage[margin=1in]{geometry}
\setlength{\emergencystretch}{3em} % prevent overfull lines
\setcounter{secnumdepth}{-\maxdimen} % remove section numbering
% Make \paragraph and \subparagraph free-standing
\ifx\paragraph\undefined\else
  \let\oldparagraph\paragraph
  \renewcommand{\paragraph}[1]{\oldparagraph{#1}\mbox{}}
\fi
\ifx\subparagraph\undefined\else
  \let\oldsubparagraph\subparagraph
  \renewcommand{\subparagraph}[1]{\oldsubparagraph{#1}\mbox{}}
\fi

\usepackage{color}
\usepackage{fancyvrb}
\newcommand{\VerbBar}{|}
\newcommand{\VERB}{\Verb[commandchars=\\\{\}]}
\DefineVerbatimEnvironment{Highlighting}{Verbatim}{commandchars=\\\{\}}
% Add ',fontsize=\small' for more characters per line
\usepackage{framed}
\definecolor{shadecolor}{RGB}{241,243,245}
\newenvironment{Shaded}{\begin{snugshade}}{\end{snugshade}}
\newcommand{\AlertTok}[1]{\textcolor[rgb]{0.68,0.00,0.00}{#1}}
\newcommand{\AnnotationTok}[1]{\textcolor[rgb]{0.37,0.37,0.37}{#1}}
\newcommand{\AttributeTok}[1]{\textcolor[rgb]{0.40,0.45,0.13}{#1}}
\newcommand{\BaseNTok}[1]{\textcolor[rgb]{0.68,0.00,0.00}{#1}}
\newcommand{\BuiltInTok}[1]{\textcolor[rgb]{0.00,0.23,0.31}{#1}}
\newcommand{\CharTok}[1]{\textcolor[rgb]{0.13,0.47,0.30}{#1}}
\newcommand{\CommentTok}[1]{\textcolor[rgb]{0.37,0.37,0.37}{#1}}
\newcommand{\CommentVarTok}[1]{\textcolor[rgb]{0.37,0.37,0.37}{\textit{#1}}}
\newcommand{\ConstantTok}[1]{\textcolor[rgb]{0.56,0.35,0.01}{#1}}
\newcommand{\ControlFlowTok}[1]{\textcolor[rgb]{0.00,0.23,0.31}{#1}}
\newcommand{\DataTypeTok}[1]{\textcolor[rgb]{0.68,0.00,0.00}{#1}}
\newcommand{\DecValTok}[1]{\textcolor[rgb]{0.68,0.00,0.00}{#1}}
\newcommand{\DocumentationTok}[1]{\textcolor[rgb]{0.37,0.37,0.37}{\textit{#1}}}
\newcommand{\ErrorTok}[1]{\textcolor[rgb]{0.68,0.00,0.00}{#1}}
\newcommand{\ExtensionTok}[1]{\textcolor[rgb]{0.00,0.23,0.31}{#1}}
\newcommand{\FloatTok}[1]{\textcolor[rgb]{0.68,0.00,0.00}{#1}}
\newcommand{\FunctionTok}[1]{\textcolor[rgb]{0.28,0.35,0.67}{#1}}
\newcommand{\ImportTok}[1]{\textcolor[rgb]{0.00,0.46,0.62}{#1}}
\newcommand{\InformationTok}[1]{\textcolor[rgb]{0.37,0.37,0.37}{#1}}
\newcommand{\KeywordTok}[1]{\textcolor[rgb]{0.00,0.23,0.31}{#1}}
\newcommand{\NormalTok}[1]{\textcolor[rgb]{0.00,0.23,0.31}{#1}}
\newcommand{\OperatorTok}[1]{\textcolor[rgb]{0.37,0.37,0.37}{#1}}
\newcommand{\OtherTok}[1]{\textcolor[rgb]{0.00,0.23,0.31}{#1}}
\newcommand{\PreprocessorTok}[1]{\textcolor[rgb]{0.68,0.00,0.00}{#1}}
\newcommand{\RegionMarkerTok}[1]{\textcolor[rgb]{0.00,0.23,0.31}{#1}}
\newcommand{\SpecialCharTok}[1]{\textcolor[rgb]{0.37,0.37,0.37}{#1}}
\newcommand{\SpecialStringTok}[1]{\textcolor[rgb]{0.13,0.47,0.30}{#1}}
\newcommand{\StringTok}[1]{\textcolor[rgb]{0.13,0.47,0.30}{#1}}
\newcommand{\VariableTok}[1]{\textcolor[rgb]{0.07,0.07,0.07}{#1}}
\newcommand{\VerbatimStringTok}[1]{\textcolor[rgb]{0.13,0.47,0.30}{#1}}
\newcommand{\WarningTok}[1]{\textcolor[rgb]{0.37,0.37,0.37}{\textit{#1}}}

\providecommand{\tightlist}{%
  \setlength{\itemsep}{0pt}\setlength{\parskip}{0pt}}\usepackage{longtable,booktabs,array}
\usepackage{calc} % for calculating minipage widths
% Correct order of tables after \paragraph or \subparagraph
\usepackage{etoolbox}
\makeatletter
\patchcmd\longtable{\par}{\if@noskipsec\mbox{}\fi\par}{}{}
\makeatother
% Allow footnotes in longtable head/foot
\IfFileExists{footnotehyper.sty}{\usepackage{footnotehyper}}{\usepackage{footnote}}
\makesavenoteenv{longtable}
\usepackage{graphicx}
\makeatletter
\def\maxwidth{\ifdim\Gin@nat@width>\linewidth\linewidth\else\Gin@nat@width\fi}
\def\maxheight{\ifdim\Gin@nat@height>\textheight\textheight\else\Gin@nat@height\fi}
\makeatother
% Scale images if necessary, so that they will not overflow the page
% margins by default, and it is still possible to overwrite the defaults
% using explicit options in \includegraphics[width, height, ...]{}
\setkeys{Gin}{width=\maxwidth,height=\maxheight,keepaspectratio}
% Set default figure placement to htbp
\makeatletter
\def\fps@figure{htbp}
\makeatother
\newlength{\cslhangindent}
\setlength{\cslhangindent}{1.5em}
\newlength{\csllabelwidth}
\setlength{\csllabelwidth}{3em}
\newlength{\cslentryspacingunit} % times entry-spacing
\setlength{\cslentryspacingunit}{\parskip}
\newenvironment{CSLReferences}[2] % #1 hanging-ident, #2 entry spacing
 {% don't indent paragraphs
  \setlength{\parindent}{0pt}
  % turn on hanging indent if param 1 is 1
  \ifodd #1
  \let\oldpar\par
  \def\par{\hangindent=\cslhangindent\oldpar}
  \fi
  % set entry spacing
  \setlength{\parskip}{#2\cslentryspacingunit}
 }%
 {}
\usepackage{calc}
\newcommand{\CSLBlock}[1]{#1\hfill\break}
\newcommand{\CSLLeftMargin}[1]{\parbox[t]{\csllabelwidth}{#1}}
\newcommand{\CSLRightInline}[1]{\parbox[t]{\linewidth - \csllabelwidth}{#1}\break}
\newcommand{\CSLIndent}[1]{\hspace{\cslhangindent}#1}

\usepackage{booktabs}
\usepackage{longtable}
\usepackage{array}
\usepackage{multirow}
\usepackage{wrapfig}
\usepackage{float}
\usepackage{colortbl}
\usepackage{pdflscape}
\usepackage{tabu}
\usepackage{threeparttable}
\usepackage{threeparttablex}
\usepackage[normalem]{ulem}
\usepackage{makecell}
\usepackage{xcolor}
\usepackage{xr}
\externaldocument{si}
\usepackage[default]{sourcesanspro}
\usepackage{sourcecodepro}
\usepackage{lineno}
\usepackage{fvextra}
\linenumbers
\DefineVerbatimEnvironment{Highlighting}{Verbatim}{breaklines,commandchars=\\\{\}}
\KOMAoption{captions}{tablesignature}
\makeatletter
\makeatother
\makeatletter
\makeatother
\makeatletter
\@ifpackageloaded{caption}{}{\usepackage{caption}}
\AtBeginDocument{%
\ifdefined\contentsname
  \renewcommand*\contentsname{Table of contents}
\else
  \newcommand\contentsname{Table of contents}
\fi
\ifdefined\listfigurename
  \renewcommand*\listfigurename{List of Figures}
\else
  \newcommand\listfigurename{List of Figures}
\fi
\ifdefined\listtablename
  \renewcommand*\listtablename{List of Tables}
\else
  \newcommand\listtablename{List of Tables}
\fi
\ifdefined\figurename
  \renewcommand*\figurename{Fig.}
\else
  \newcommand\figurename{Fig.}
\fi
\ifdefined\tablename
  \renewcommand*\tablename{Table}
\else
  \newcommand\tablename{Table}
\fi
}
\@ifpackageloaded{float}{}{\usepackage{float}}
\floatstyle{ruled}
\@ifundefined{c@chapter}{\newfloat{codelisting}{h}{lop}}{\newfloat{codelisting}{h}{lop}[chapter]}
\floatname{codelisting}{Listing}
\newcommand*\listoflistings{\listof{codelisting}{List of Listings}}
\makeatother
\makeatletter
\@ifpackageloaded{caption}{}{\usepackage{caption}}
\@ifpackageloaded{subcaption}{}{\usepackage{subcaption}}
\makeatother
\makeatletter
\@ifpackageloaded{tcolorbox}{}{\usepackage[many]{tcolorbox}}
\makeatother
\makeatletter
\@ifundefined{shadecolor}{\definecolor{shadecolor}{rgb}{.97, .97, .97}}
\makeatother
\makeatletter
\makeatother
\ifLuaTeX
  \usepackage{selnolig}  % disable illegal ligatures
\fi
\IfFileExists{bookmark.sty}{\usepackage{bookmark}}{\usepackage{hyperref}}
\IfFileExists{xurl.sty}{\usepackage{xurl}}{} % add URL line breaks if available
\urlstyle{same} % disable monospaced font for URLs
\hypersetup{
  colorlinks=true,
  linkcolor={blue},
  filecolor={Maroon},
  citecolor={Blue},
  urlcolor={Blue},
  pdfcreator={LaTeX via pandoc}}

\author{}
\date{}

\begin{document}
\ifdefined\Shaded\renewenvironment{Shaded}{\begin{tcolorbox}[borderline west={3pt}{0pt}{shadecolor}, sharp corners, enhanced, boxrule=0pt, breakable, interior hidden, frame hidden]}{\end{tcolorbox}}\fi

\textbf{Title: Title of your manuscript}

\[ \]

First Author\textsuperscript{1,2}, Second Author\textsuperscript{2}, and
Third Author\textsuperscript{3}

\[ \]

\textsuperscript{1} First Affiliation

\textsuperscript{2} Second Affiliation

\textsuperscript{3} Third Affiliation

Your abstract.

\hypertarget{introduction}{%
\section{Introduction}\label{introduction}}

This is a manuscript template for Quarto markdown that uses R packages
\texttt{targets} (\protect\hyperlink{ref-Landau2021}{Landau 2021}) and
\texttt{stantargets}. \texttt{R/functions.R} contains R codes that I
often use.

\hypertarget{examples}{%
\section{Examples}\label{examples}}

\hypertarget{equations}{%
\subsection{Equations}\label{equations}}

A centered parameterization of the Eight Schools model (Eq.~\ref{eq-cp};
Gelman et al. (\protect\hyperlink{ref-Gelman2013}{2013})).

\begin{equation}\protect\hypertarget{eq-cp}{}{
\begin{aligned}
\mu &\sim N(0,5) \\
\tau &\sim HalfCauchy(0,5) \\
\theta_j &\sim N(\mu, \tau) \\
y_j &\sim N(\theta_j, \sigma_j)
\end{aligned}
}\label{eq-cp}\end{equation}

You can group multiple lines of equations to a single equation label.

In a non-centered parameterization of the Eq.~\ref{eq-cp}, we fit latent
Gaussian variables instead of directly estimating \(\theta_j\):

\begin{equation}\protect\hypertarget{eq-latent}{}{
\tilde{\theta}_j \sim N(0, 1)
}\label{eq-latent}\end{equation}

\begin{equation}\protect\hypertarget{eq-ncp}{}{
\theta_j = \mu + \tau \tilde{\theta}_j.
}\label{eq-ncp}\end{equation}

You can label each line too.

The half-cauchy distribution in the Eq.~\ref{eq-cp} can be further
rewritten as following:

\[
\begin{aligned}
\tilde{\tau} \sim U(0, \pi / 2) \\
\tau = 5 tan(\tilde{\tau})
\end{aligned}
\]

You can also write equations without labels.

Source codes can be loaded and printed, which may be useful for
supporting information.

\begin{Shaded}
\begin{Highlighting}[]
\KeywordTok{data}\NormalTok{ \{}
  \DataTypeTok{int}\NormalTok{\textless{}}\KeywordTok{lower}\NormalTok{=}\DecValTok{0}\NormalTok{\textgreater{} J;}
  \DataTypeTok{vector}\NormalTok{[J] y;}
  \DataTypeTok{vector}\NormalTok{\textless{}}\KeywordTok{lower}\NormalTok{=}\DecValTok{0}\NormalTok{\textgreater{}[J] sigma;}
\NormalTok{\}}

\KeywordTok{parameters}\NormalTok{ \{}
  \DataTypeTok{real}\NormalTok{ mu;}
  \DataTypeTok{real}\NormalTok{\textless{}}\KeywordTok{lower}\NormalTok{=}\DecValTok{0}\NormalTok{,}\KeywordTok{upper}\NormalTok{=pi()/}\DecValTok{2}\NormalTok{\textgreater{} tau\_unif;}
  \DataTypeTok{real}\NormalTok{ theta\_tilde[J];}
\NormalTok{\}}

\KeywordTok{transformed parameters}\NormalTok{ \{}
  \DataTypeTok{real}\NormalTok{\textless{}}\KeywordTok{lower}\NormalTok{=}\DecValTok{0}\NormalTok{\textgreater{} tau;}
  \DataTypeTok{real}\NormalTok{ theta[J];}
\NormalTok{  tau = }\DecValTok{5}\NormalTok{ * tan(tau\_unif);}
  \ControlFlowTok{for}\NormalTok{ (j }\ControlFlowTok{in} \DecValTok{1}\NormalTok{:J)}
\NormalTok{    theta[j] = mu + tau * theta\_tilde[j];}
\NormalTok{\}}

\KeywordTok{model}\NormalTok{ \{}
\NormalTok{  mu \textasciitilde{} normal(}\DecValTok{0}\NormalTok{, }\DecValTok{5}\NormalTok{);}
\NormalTok{  theta\_tilde \textasciitilde{} std\_normal();}
\NormalTok{  y \textasciitilde{} normal(theta, sigma);}
\NormalTok{\}}

\KeywordTok{generated quantities}\NormalTok{ \{}
  \DataTypeTok{vector}\NormalTok{[J] log\_lik;}
  \ControlFlowTok{for}\NormalTok{ (j }\ControlFlowTok{in} \DecValTok{1}\NormalTok{:J) log\_lik[j] = normal\_lpdf(y[j] | theta[j], sigma[j]);}
\NormalTok{\}}
\end{Highlighting}
\end{Shaded}

\hypertarget{tables}{%
\subsection{Tables}\label{tables}}

It's easier to use \texttt{kableExtra} than manually writing markdown
tables. Here is the R code chunk to produce
Table~\ref{tbl-eight_schools}.

\begin{Shaded}
\begin{Highlighting}[]
\NormalTok{schools\_data }\OtherTok{\textless{}{-}} \FunctionTok{tibble}\NormalTok{(}
  \AttributeTok{School =}\NormalTok{ LETTERS[}\DecValTok{1}\SpecialCharTok{:}\DecValTok{8}\NormalTok{],}
  \StringTok{\textasciigrave{}}\AttributeTok{Estimated treatment effect, $y\_j$}\StringTok{\textasciigrave{}}  \OtherTok{=} \FunctionTok{c}\NormalTok{(}\DecValTok{28}\NormalTok{, }\DecValTok{8}\NormalTok{, }\SpecialCharTok{{-}}\DecValTok{3}\NormalTok{, }\DecValTok{7}\NormalTok{, }\SpecialCharTok{{-}}\DecValTok{1}\NormalTok{, }\DecValTok{1}\NormalTok{, }\DecValTok{18}\NormalTok{, }\DecValTok{12}\NormalTok{),}
  \StringTok{\textasciigrave{}}\AttributeTok{Standard error of effect estimate, $}\SpecialCharTok{\textbackslash{}\textbackslash{}}\AttributeTok{sigma\_j$}\StringTok{\textasciigrave{}} \OtherTok{=} \FunctionTok{c}\NormalTok{(}\DecValTok{15}\NormalTok{, }\DecValTok{10}\NormalTok{, }\DecValTok{16}\NormalTok{, }\DecValTok{11}\NormalTok{, }\DecValTok{9}\NormalTok{, }\DecValTok{11}\NormalTok{, }\DecValTok{10}\NormalTok{, }\DecValTok{18}\NormalTok{))}

\NormalTok{schools\_data }\SpecialCharTok{|\textgreater{}}
  \FunctionTok{kbl}\NormalTok{(}\AttributeTok{booktabs =} \ConstantTok{TRUE}\NormalTok{, }\AttributeTok{escape =} \ConstantTok{FALSE}\NormalTok{) }\SpecialCharTok{|\textgreater{}}
    \FunctionTok{kable\_styling}\NormalTok{(}\AttributeTok{latex\_options =} \StringTok{"striped"}\NormalTok{)}
\end{Highlighting}
\end{Shaded}

\hypertarget{figures}{%
\subsection{Figures}\label{figures}}

I currently prefer to use \texttt{!{[}{]}()} to insert images
(Fig.~\ref{fig-theta_tau}; Fig.~\ref{fig-theta_intervals}) rather than
using R code chunks. It's easy for cross-references and putting greek
letters. The path for Fig.~\ref{fig-theta_tau} can be
\texttt{figs/theta\_tau\_line.png} or
\texttt{..figs/theta\_tau\_line.png} depending on where you are working.
When you run \texttt{make} (i.e., Makefile), the first one is the
correct path. When you use \texttt{render} in VSCode, the second one is
correct. Using \texttt{here::here} is useful to specify the path to
figures.

\hypertarget{cross-reference-from-different-files}{%
\subsection{Cross-reference from different
files}\label{cross-reference-from-different-files}}

We can also do cross-reference from different files, which is useful to
refer figures and tables in supporting information.

Fig. S\texttt{\textbackslash{}ref\{fig-hist\}}.

The above command will render the following

Fig. S\ref{fig-hist}.

You need these in the YAML.

\begin{verbatim}
\usepackage{xr}
\externaldocument{si}
\end{verbatim}

This only works on LaTeX.

\hypertarget{parameterized-text}{%
\subsection{Parameterized text}\label{parameterized-text}}

The posterior median of treatment effect for school A (\(\theta_1\)) is
\texttt{\textasciigrave{}r\ get\_post\_para(para,\ "theta{[}1{]}",\ "q50")\textasciigrave{}}
with the 95\% credible interval of {[}
\texttt{\textasciigrave{}\ r\ get\_post\_para(para,\ "theta{[}1{]}",\ "q2.5")\textasciigrave{}}
,
\texttt{\textasciigrave{}\ r\ get\_post\_para(para,\ "theta{[}1{]}",\ "q97.5",\ digits\ =\ 1,\ nsmall\ =\ 1)\textasciigrave{}}{]}.

The above text will be rendered as following:

The posterior median of treatment effect for school A (\(\theta_1\)) is
5.68 with the 95\% credible interval of {[} -3.10 , 19.2{]}.

\hypertarget{references}{%
\section{References}\label{references}}

\hypertarget{refs}{}
\begin{CSLReferences}{1}{0}
\leavevmode\vadjust pre{\hypertarget{ref-Gelman2013}{}}%
Gelman, A., J. B. Carlin, H. S. Stern, D. B. Dunson, A. Vehtari, and D.
B. Rubin. 2013. Bayesian {Data Analysis}, {Third Edition}. {Chapman \&
Hall/CRC}, {Boca Raton, FL, USA.}

\leavevmode\vadjust pre{\hypertarget{ref-Landau2021}{}}%
Landau, W. M. 2021. \href{https://doi.org/10.21105/joss.02959}{The
targets {R} package: A dynamic {Make-like} function-oriented pipeline
toolkit for reproducibility and high-performance computing}. Journal of
Open Source Software 6:2959.

\end{CSLReferences}

\newpage

\hypertarget{tables-1}{%
\section{Tables}\label{tables-1}}

\begin{table}

\end{table}

\hypertarget{tbl-eight_schools}{}
\begin{table}[h!]
\caption{\label{tbl-eight_schools}Observed effects of special preparation on SAT-V scores in eight
randomized experiments }\tabularnewline


\begin{tabular}[t]{lrr}
\toprule
School & Estimated treatment effect, $y_j$ & Standard error of effect estimate, $\sigma_j$\\
\midrule
\cellcolor{gray!6}{A} & \cellcolor{gray!6}{28} & \cellcolor{gray!6}{15}\\
B & 8 & 10\\
\cellcolor{gray!6}{C} & \cellcolor{gray!6}{-3} & \cellcolor{gray!6}{16}\\
D & 7 & 11\\
\cellcolor{gray!6}{E} & \cellcolor{gray!6}{-1} & \cellcolor{gray!6}{9}\\
\addlinespace
F & 1 & 11\\
\cellcolor{gray!6}{G} & \cellcolor{gray!6}{18} & \cellcolor{gray!6}{10}\\
H & 12 & 18\\
\bottomrule
\end{tabular}
\end{table}

Estimates are based on separate analyses for the eight experiments
(\protect\hyperlink{ref-Gelman2013}{Gelman et al. 2013}).

\begin{quote}
\begin{itemize}
\tightlist
\item
  I don't know how to add greek letters in the caption but at least I
  can put greek letters outside the caption (e.g., \(\theta_j\)).
\end{itemize}
\end{quote}

\begin{quote}
\begin{itemize}
\tightlist
\item
  There is a bug for table cross-reference
  \href{https://github.com/quarto-dev/quarto-cli/discussions/2264\#discussioncomment-3541972}{(Section
  titles appear after tables (PDF) \#2264)}.
\end{itemize}
\end{quote}

\newpage

\hypertarget{figures-1}{%
\section{Figures}\label{figures-1}}

\begin{figure}

{\centering \includegraphics{/home/mattocci/project-template/figs/theta_tau_line.png}

}

\caption{\label{fig-theta_tau}The famous plot from Gelman et al.
(\protect\hyperlink{ref-Gelman2013}{2013}). Conditional posterior means
of treatment effects, \(E(\theta_j | \tau, y)\) is plotted against the
between school standard deviation \(\tau\). The stan model was fitted
using dynamic branches.}

\end{figure}

\newpage

\begin{figure}

{\centering \includegraphics{/home/mattocci/project-template/figs/theta_intervals.png}

}

\caption{\label{fig-theta_intervals}The posterior means, 50\% and 90\%
credible intervals of treatment effects (\(\theta_j\)).}

\end{figure}



\end{document}
